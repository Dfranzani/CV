%!TEX TS-program = xelatex
%!TEX encoding = UTF-8 Unicode
% Awesome CV LaTeX Template for CV/Resume
%
% This template has been downloaded from:
% https://github.com/posquit0/Awesome-CV
%
% Author:
% Claud D. Park <posquit0.bj@gmail.com>
% http://www.posquit0.com
%
%
% Adapted to be an Rmarkdown template by Mitchell O'Hara-Wild
% 23 November 2018
%
% Template license:
% CC BY-SA 4.0 (https://creativecommons.org/licenses/by-sa/4.0/)
%
%-------------------------------------------------------------------------------
% CONFIGURATIONS
%-------------------------------------------------------------------------------
% A4 paper size by default, use 'letterpaper' for US letter
\documentclass[11pt,a4paper,]{awesome-cv}

% Configure page margins with geometry
\usepackage{geometry}
\geometry{left=1.4cm, top=.8cm, right=1.4cm, bottom=1.8cm, footskip=.5cm}


% Specify the location of the included fonts
\fontdir[fonts/]

% Color for highlights
% Awesome Colors: awesome-emerald, awesome-skyblue, awesome-red, awesome-pink, awesome-orange
%                 awesome-nephritis, awesome-concrete, awesome-darknight

\definecolor{awesome}{HTML}{414141}

% Colors for text
% Uncomment if you would like to specify your own color
% \definecolor{darktext}{HTML}{414141}
% \definecolor{text}{HTML}{333333}
% \definecolor{graytext}{HTML}{5D5D5D}
% \definecolor{lighttext}{HTML}{999999}

% Set false if you don't want to highlight section with awesome color
\setbool{acvSectionColorHighlight}{true}

% If you would like to change the social information separator from a pipe (|) to something else
\renewcommand{\acvHeaderSocialSep}{\quad\textbar\quad}

\def\endfirstpage{\newpage}

%-------------------------------------------------------------------------------
%	PERSONAL INFORMATION
%	Comment any of the lines below if they are not required
%-------------------------------------------------------------------------------
% Available options: circle|rectangle,edge/noedge,left/right

\name{}{Daniel Franzani Cerda}


\mobile{+56 9 6244 8359}
\email{\href{mailto:dfranzani@gmail.com}{\nolinkurl{dfranzani@gmail.com}}}
\homepage{dfranzani.github.io/website}
\github{dfranzani}
\linkedin{dfranzani}

% \gitlab{gitlab-id}
% \stackoverflow{SO-id}{SO-name}
% \skype{skype-id}
% \reddit{reddit-id}


\usepackage{booktabs}

\providecommand{\tightlist}{%
	\setlength{\itemsep}{0pt}\setlength{\parskip}{0pt}}

%------------------------------------------------------------------------------



% Pandoc CSL macros

\begin{document}

% Print the header with above personal informations
% Give optional argument to change alignment(C: center, L: left, R: right)
\makecvheader

% Print the footer with 3 arguments(<left>, <center>, <right>)
% Leave any of these blank if they are not needed
% 2019-02-14 Chris Umphlett - add flexibility to the document name in footer, rather than have it be static Curriculum Vitae
\makecvfooter
  {mayo 2024}
    { Daniel Franzani Cerda~~~·~~~Curriculum Vitae}
  {\thepage}


%-------------------------------------------------------------------------------
%	CV/RESUME CONTENT
%	Each section is imported separately, open each file in turn to modify content
%------------------------------------------------------------------------------



\section{Experiencia Profesional}\label{experiencia-profesional}

\begin{cventries}
    \cventry{Universidad Diego Portales}{Coordinador de línea Estadística}{}{Agosto 2022 - Presente}{\begin{cvitems}
\item Académico Jornada Parcial, Departamento de Matemáticas y Estadística, Facultad de Administración y Economía.
\item Elaboración de los programas de los cursos de estadística de la Facultad de Administración y Economía.
\item Responsable del diseño, implementación y lanzamiento del material de eseñanza a través de páginas web estáticas desarrolladas mediante \textit{Bookdown} y  \textit{Quarto}, incorporando el uso de Google Colab R.
\item Responsable del seguimiento de la enseñanza de los contenidos impartidos por los distintos docentes en los cursos de estadística de la Facultad de Administración y Economía.
\end{cvitems}}
    \cventry{Universidad Tecnológina Metropolitana}{Profesional Estadístico}{}{Novimebre 2020 - Noviembre 2022}{\begin{cvitems}
\item Profesional estadístico del Departamento de Seguimiento a la Progresión de los Estudiantes.
\item Liderazgo demostrado como enlace interdepartamental entres distintos equipos de la universidad, para la socialización de los resultados de progresión académica, definición de métricas y disponibilidad de datos.
\item Diseño e implementarción de flujos de trabajo ETL para BigQuery para optimizar y mejorar los procesos de ingesta, accesibilidad e integridad de los datos para su posterior análisis.
\item Creación y automatización de informes (R Markdown) de datos institucionales, modelos estadísticos (Modelos Lineales Generalizados, Modelos de Supervivencia, KNN, entre otros) y paneles de control de datos personalizados (Power BI), para ayudar a equipos en la toma de decisiones para la retención estudiantil.
\item Asesor en el pronóstico y monitoreo de indicadores clave de desempeño a nivel institucional y en la construcción del primer Mecanismo de Alerta Académica Institucional enfocado en la detección de deserción estudiantil en tiempo real (\textit{LightGBM}), utilizando conjuntos de datos socioeconómicos y demográficos.
\end{cvitems}}
    \cventry{TecPar}{Científico de Datos}{(Independiente)}{Agosto 2019 - Presente}{\begin{cvitems}
\item Utilización de scripts bash de R y GNU-Linux para analizar y explorar conjuntos de datos complejos para diversos proyectos de investigación.
\item Creación de scripts personalizados para crear piplines ETL escalables mediante Spark y SQL (DuckDB), con el fin de mejorar las capacidades de procesamiento de datos y el despliegue de modelos en ambientes virtuales locales (R, Python).
\item Desarrollo de modelos y simulaciones (Modelos Lineales Generalizados) basados en datos de panel para probar efectos semi-cuasi experimentales de indicadores macro económicos.
\item Desarrollo de chatbots internos especializados en generar informes automatizados (R Sweave), reduciendo el tiempo de gestión de ejecutivos.
\item Ajuste y monitoreo de modelos descriptivos y predictivos de Machine Learning y Deep Learning para distintos proyectos, tales como, el consumo eléctrico en hogares a través del tiempo (SARIMA, LSTM), riesgo de incumplimiento de pagos de servicios privados (Modelos Lineales Generalizados), detección de patrones del índice de vegetación NDVI para los tiempos de cosecha (Redes Neuronales Convolucionales y estimación de kernel), análisis de consumo de bienes de lujo bajo factores religiosos y sociales (Modelo de Ecuaciones Estructurales) y análisis del consumo de agua en hogares (análisis no-paramétrico).
\item Diseño de una arquitectura de Deep Learning (Red Neuronal Convolucional) para el análisis de vídeos de vigilancia en tiempo real, enfocado en la detección de delitos de robo de autos, mediante YOLOv5 y Heurísticas de clasificación.
\end{cvitems}}
    \cventry{SGS}{Profesional Estadístico}{}{Octubre 2021 - Diciembre 2022}{\begin{cvitems}
\item Responsable experto para el proyecto de Economía Circular - LATAM Airlines en cuestiones técnicas, que abarcan la consolidación de bases de datos asociadas a la generación de residuos de los distintos vuelos, elaboración de indicadores operacionales, elaboración de paneles de visualización en Power BI y la automatización de reportes (R Markdown) para la socialización de resultados.
\end{cvitems}}
    \cventry{Banco Santander}{Auditoría Interna}{}{Enero 2019 - Febrero 2019}{\begin{cvitems}
\item Monitoreo de modelos de \textit{scoring}, \textit{rating}, parámetros de capital y provisiones.
\item Elaboración de indicadores de desempeño y discriminación de modelos de \textit{scoring}, \textit{rating}, parámetros de capital y provisiones.
\end{cvitems}}
\end{cventries}

\section{Proyectos de desarrollo}\label{proyectos-de-desarrollo}

\begin{cvhonors}
    \cvhonor{}{\textbf{Tablyzer}: Desarrollador \newline Librería en R: Infraestructura para evaluar las inconsistencias en el cruce de bases de datos relacionales. \textit{Estado: privada, en desarrollo}.}{}{2023}
    \cvhonor{}{\textbf{\href{https://github.com/Dfranzani/Mglm}{Mglm}}: Desarrollador principal \newline Librería en R: Infraestructura para evaluar el desempeño de Modelos Lineales Generalizados. \textit{Estado: pública}.}{}{2019}
\end{cvhonors}

\newpage{}

\section{Experiencia Docente}\label{experiencia-docente}

\subsection{Pregrado}\label{pregrado}

\vspace{-0.6cm}\begin{cventries}
    \cventry{}{Universidad Diego Portales \vspace{-0.6cm}}{Marzo 2022 - Presente \vspace{-0.6cm}}{}{\begin{cvitems}
\item Estadística I (ICO09212)/Estadística Descriptiva (ICG3464)/Estadística (AUD8054): tópicos básicos de estadística, gráficos descriptivos, probabilidades, variables aleatorias discretas y continuas (función de densidad de probabilidad, función de distribución acumulada, principales distribuciones), ditribuciones muestrales.
\item Estadística II (ICO09221)/Inferencia Estadística (ICG3465): intervalos de confianza, pruebas de hipótesis, regresión lineal simple y múltiple, ANOVA.
\item Matemáticas III (AUD8053): derivadas parciales, derivación parcial implícita, optimización de funciones multivariadas, análisis marginal, elasticidad de demanda, matrices, sistema de ecuaciones lineales.
\end{cvitems}}
\end{cventries}\vspace{-0.6cm}\begin{cventries}
    \cventry{}{Universidad Católica Silva Henríquez \vspace{-0.6cm}}{Agosto 2019 - Diciembre 2021 \vspace{-0.6cm}}{}{\begin{cvitems}
\item Estadística I (DMPM52): tópicos básicos de estadística, elementos de probabilidad, variables aleatorias discretas (función de densidad conjunta, valores esperados conjuntos marginales y condicionales, función generadora de momentos, modelos simples y bivariados).
\item Estadística II (DMPM63): variables aleatorias continuas (función de densidad conjunta, valores esperados conjuntos marginales y condicionales, transformaciones lineales, función generadora de momentos, modelos simples y bivariados), muestra y distribuciones muestrales, estimación de parámetros, pruebas de hipótesis.
\item Matemática II (RCA203): tasa de variación de una función, cálculo diferencial en una variable.
\end{cvitems}}
\end{cventries}

\subsection{Postgrado}\label{postgrado}

\begin{cventries}
    \cventry{}{Universidad Diego Portales}{Noviembre 2022 - Presente}{}{\begin{cvitems}
\item Asesor de Métodos Cuantitativos para el curso de Seminario de grado del Magíster en Dirección de Marketing. Área: comportamiento del consumidor. Técnicas: Modelo de Ecuaciones Estructurales y Modelos Lineales Generalizados. Las tesis asesoradas son las siguientes:
\end{cvitems}}
\end{cventries}

\vspace{-0.3cm} \begingroup \footnotesize

\begin{itemize}
    \item[--] (2023-2) Estrategia de Marketing Onmicanalidad: Investigación de empresas en Chile que convergen las técnicas de Marketing tradicionales y digitales.
    \item[--] (2023-2) Modelo de la aceptación tecnológica: Una comparación de las intensiones de compra en línea entre las generaciones chilenas de Baby Boomers y Millennials.
    \item[--] (2023-2) Autoconcepto y presión de pares: Un análisis de sus efectos en el Materialismo y la Actitud frente al lujo.
    \item[--] (2023-1) \href{https://repositoriobiblioteca.udp.cl/TD002861.pdf#pagemode=thumbs}{Percepción de los televidentes sobre la reputación coporativa de los canales de televisión en Chile.}
    \item[--] (2022-2) El impacto del autoconcepto en relación con las actitudes hacia el lujo: El efecto del materialismo y motivaciones sociales para consumir.
    \item[--] (2022-2) Factores que afectan a la percepción de las personas mayores en Chile en cuanto al uso de teléfonos inteligentes.
  \end{itemize}
  \endgroup
  \vspace{-0.7cm}

\begin{cventries}
    \cventry{}{}{}{}{\begin{cvitems}
\item Tesis asesoradas en otros programas de magíster:
\end{cvitems}}
\end{cventries}

\vspace{-0.3cm} \begingroup \footnotesize

\begin{itemize}
    \item[--] (2023-2, Magíster en Finanzas) El impacto de la descapitalización de los fondos de pensiones en el costo de la deuda corporativa: El caso de Chile.
    \item[--] (2023-2, Magíster en Finanzas) Existencia del efecto tamaño en las acciones chilenas.
    \item[--] (2023-2, Magíster en Personas y Organizaciones) Liderazgo Laissez-Faire y consumo de psicotrópicos en trabajadores de área de salud en Chile, considerando dimensiones de género.
    \item[--] (2022-2, Magíster en Negocios Digitales) \href{https://repositoriobiblioteca.udp.cl/TD002496.pdf#pagemode=thumbs}{Análisis de trayectorias de futbolistas extranjeros : relación de las trayectorias con el desempeño de los equipos de la Serie A}.
  \end{itemize}
  \endgroup

\section{Educación}\label{educaciuxf3n}

\begin{cventries}
    \cventry{Magíster en Inteligencia Artificial}{Universidad Adolfo Ibáñez}{Santiago, Chile}{Diciembre 2022}{}\vspace{-4.0mm}
\end{cventries}\vspace{-0.5cm}\begin{cventries}
    \cventry{Magíster en Estadística}{Pontificia Universidad Católica de Chile}{Santiago, Chile}{Diciembre 2019}{}\vspace{-4.0mm}
\end{cventries}\vspace{-0.5cm}\begin{cventries}
    \cventry{Licenciatura en Educación/Pedagogía en Matemática}{Universidad Católica Silva Henríquez}{Santiago, Chile}{Julio 2017}{}\vspace{-4.0mm}
\end{cventries}



\end{document}
